\newglossaryentry{frontend}
{
    name=frontend,
    description={Se encarga de la parte del software que interactúa con el usuario y que el usuario puede ver.}
}
\newglossaryentry{backend}
{
    name=backend,
    description={Representa la capa de acceso de datos.}
}

\newacronym{ajax}{AJAX}{Asynchronous JavaScript And XML (JavaScript asíncrono y XML)}
\newacronym{spa}{SPA}{Single Page Application (Aplicación de una sola página)}
\newacronym{epel}{EPEL}{Extra Packages for Enterprise Linux (Paquetes Extra para Linux Empresarial)}
\newacronym{jit}{JIT}{Just In Time (Justo A Tiempo)}
\newacronym{npm}{NPM}{Node Package Manager (Administrados de Paquetes Node)}
\newacronym{dom}{DOM}{Document Object Model (Modelo de Objeto de Documento)}
\newacronym{rdbms}{RDBMS}{Relational Database Management System (Sistema de Gestión de Bases de Datos Relacionales)}
\newacronym{rest}{REST}{Representational State Transfer (Transferencia de Estado Representacional)}
\newacronym{api}{API}{Application Programming Interface (Interfaz de Programación de Aplicaciones)}
\newacronym{crud}{CRUD}{Create, Read, Update, and Delete (Crear, Leer, Actualizar y Borrar)}

\newglossaryentry{mixin}
{
    name=mixin,
    plural=mixins,
    description={En los lenguajes de programación orientada a objetos, un mixin es una clase que ofrece cierta funcionalidad para ser heredada por una subclase, pero no está ideada para ser autónoma}
}

\newglossaryentry{widget}
{
    name=widget,
    plural=widgets,
    description={Elemento de una interfaz gráfica de usuario (GUI) que muestra información o proporciona una forma específica para que un usuario interactúe con el sistema operativo o una aplicación}
}

\newglossaryentry{ionic}
{
    name=ionic,
    description={SDK completo de código abierto para el desarrollo de aplicaciones móviles híbridas}
}
\newglossaryentry{token}
{
    name=token,
    description={Un objeto del sistema que representa el sujeto de las operaciones de control de acceso}
}
\newglossaryentry{websocket}
{
    name=websocket,
    plural=websockets,
    description={Protocolo de comunicaciones informáticas que proporciona canales de comunicación dúplex completo a través de una única conexión TCP}
}
\newglossaryentry{axios}
{
    name=axios,
    description={Cliente HTTP basado en promesas para el navegador y Node.js}
}
\newglossaryentry{cookie}
{
    name=cookie,
    plural=cookies,
    description={Es una pequeña pieza de datos enviada desde un sitio web y almacenada en la computadora del usuario por el navegador web del usuario mientras el usuario está navegando}
}
\newglossaryentry{daemon}
{
    name=daemon,
    description={En los sistemas operativos de computadora multitarea, un daemon es un programa informático que se ejecuta como un proceso en segundo plano}
}
\newglossaryentry{escalabilidad}
{
    name=escalabilidad,
    description={Es un anglicismo que describe la capacidad de un negocio o sistema de crecer en magnitud}
}
\newglossaryentry{framework}
{
    name=framework,
    plural=frameworks,
    description={Un framework, entorno de trabajo​ o marco de trabajo​ es un conjunto estandarizado de conceptos, prácticas y criterios para enfocar un tipo de problemática particular que sirve como referencia, para enfrentar y resolver nuevos problemas de índole similar}
}
\newglossaryentry{hardware}
{
    name=hardware,
    description={Se refiere a las partes físicas, tangibles, de un sistema informático}
}
\newglossaryentry{software}
{
    name=software,
    description={Soporte lógico de un sistema informático, que comprende el conjunto de los componentes lógicos necesarios que hacen posible la realización de tareas específicas}
}
\newglossaryentry{holístico}
{
    name=holístico,
    description={Del todo o que considera algo como un todo}
}
\newglossaryentry{hook}
{
    name=hook,
    plural=hooks,
    description={Datos y comandos ejecutables enviados de una aplicación a otra a través de HTTP}
}
\newglossaryentry{isomorfico}
{
    name=isomórfico,
    plural=isomórficos,
    description={El isomorfismo aplicado al desarrollo web significa renderizar páginas tanto en el lado del servidor como del cliente}
}
\newglossaryentry{kernel}
{
    name=kernel,
    description={Programa de computadora que es el núcleo del sistema operativo de una computadora, con control completo sobre todo en el sistema}
}
\newglossaryentry{minificacion}
{
    name=minificación,
    description={el proceso mediante el cual se eliminan datos innecesarios o redundantes de un recurso sin que se vea afectada la forma en que los navegadores lo procesan}
}
\newglossaryentry{plugin}
{
    name=plugin,
    plural=plugins,
    description={Es un componente de software que agrega una característica específica a un programa de computadora existente}
}
\newglossaryentry{renderizado}
{
    name=renderizado,
    description={Proceso de renderizar}
}
\newglossaryentry{renderizar}
{
    name=renderizar,
    description={Proceso de generar un modelo visual a partir de una información dada}
}
\newglossaryentry{script}
{
    name=script,
    description={Lista de comandos que ejecuta un determinado programa o motor de secuencias de comandos}
}
\newglossaryentry{singleton}
{
    name=singleton,
    description={Patrón de diseño de software que restringe la creación de instancias de una clase a una instancia única}
}
\newglossaryentry{tablet}
{
    name=tablet,
    description={Computadora portátil de mayor tamaño que un teléfono inteligente o un PDA, se trata de una sola pieza que integra una pantalla táctil (sencilla o multitáctil) que emite luz y con la que se interactúa primariamente con los dedos}
}