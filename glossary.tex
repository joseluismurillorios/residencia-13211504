\newglossaryentry{frontend}
{
    name=frontend,
    description={Se encarga de la parte del software que interactúa con el usuario y que el usuario puede ver.}
}
\newglossaryentry{backend}
{
    name=backend,
    description={Representa la capa de acceso de datos.}
}

\newacronym{ajax}{AJAX}{Asynchronous JavaScript And XML (JavaScript asíncrono y XML)}
 
    % Application,
    % Atomic,
    % Authentication,
    % Bolt,
    % Chrome,
    % Claims,
    % Cloud,
    % Commerce,
    % Custom,
    % Database,
    % Design,
    % Ember,
    % Engine,
    % FERN,
    % Firebase,
    % Firestore,
    % Ionic,
    % Just,
    % LAMP,
    % MERN,
    % Manager,
    % Model,
    % NOSQL,
    % Object,
    % Package,
    % Page,
    % Preprocesamiento,
    % Provider,
    % Query,
    % RDBMS,
    % RDMS,
    % REST,
    % React,
    % Reducer,
    % Redux,
    % Rosesland,
    % Shopify,
    % Socket,
    % Store,
    % Storefront,
    % Token,
    % Websockets,
    % actions,
    % admin,
    % autenticación,
    % axios,
    % back,
    % bundle,
    % carrito,
    % conceptualizar,
    % cookies,
    % delete,
    % desinstalar,
    % elaboracion,
    % empaquetarlos,
    % enrutadores,
    % enrutamiento,
    % escalabilidad,
    % estratégicamente,
    % framework,
    % frameworks,
    % front,
    % holístico,
    % inicializar,
    % isomórficos,
    % loader,
    % loaders,
    % metadatos,
    % microempresas,
    % minificación,
    % nano,
    % optimizaciones,
    % optimización,
    % payload,
    % pligins,
    % plugins,
    % preset,
    % props,
    % reducers,
    % renderizado,
    % renderizar,
    % replicación,
    % reutilizable,
    % reutilizables,
    % reutilización,
    % root,
    % router,
    % script,
    % semiestructurados,
    % singleton,
    % subcomponentes,
    % tablet,
    % tokens,
    % update,
    % versionador,
    % webhook,
    % webhooks