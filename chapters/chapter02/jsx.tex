\subsection{ES5/ES6}
ES5 (ES significa ECMAScript) es básicamente `JavaScript normal'. La quinta actualización de JavaScript, ES5 se finalizó en 2009. Ha sido compatible con todos los principales navegadores durante muchos años. \\[0.8cm]
ES6 es una nueva versión de JavaScript que agrega algunas buenas adiciones sintácticas y funcionales. Se finalizó en 2015. ES6 es casi totalmente compatible con todos los navegadores principales. Pero pasará algún tiempo hasta que las versiones anteriores de los navegadores web estén fuera de uso. Por ejemplo, Internet Eyplorer 11 no es compatible con ES6, pero tiene aproximadamente el 8\% de uso de mercado de navegadores.\\[0.8cm]
Para aprovechar los beneficios de ES6 hoy, se tienen que hacer que hacer algunos procedimientos para que funcione en tantos exploradores de internet como sea posible:
\begin{enumerate}
  \item Se debe preprocesar el código para que una gama más amplia de navegadores entiendan nuestro JavaScript. Esto significa convertir ES6 JavaScript en ES5 JavaScript.
  \item Tenemos que incluir un `shim' o `polyfill' que brinde una funcionalidad adicional agregada en ES6 que un navegador puede o no tener.
\end{enumerate}
\subsubsection{JSX}
JSX es una extensión de sintaxis similar a XML para ECMAScript sin ninguna semántica definida. NO está destinado a ser implementado por motores o navegadores. NO es una propuesta para incorporar JSX en la propia especificación ECMAScript. Está destinado a ser utilizado por varios preprocesadores (transpiladores) para transformar estos \glspl{token} en ECMAScript estándar.
\subsubsection{BabelJS}
BabelJS es un transpilador de JavaScript que transpila nuevas características ES6 al antiguo estándar ES5. Con esto, las funciones se pueden ejecutar en navegadores antiguos y nuevos, sin problemas. \\[0.8cm]
\textbf{El preprocesador BabelJS} convierte la sintaxis de JavaScript moderno en un formulario, que los navegadores más antiguos pueden entender fácilmente. Por ejemplo, const y let se convertirán en var, la función flecha se convierte en una función normal manteniendo la funcionalidad igual en ambos casos. \\[0.8cm]

\lstinputlisting[style=ES6, caption=Ejemplo de sintaxis ES5 y ES6]{code/es5-es6.js}

\subsubsection{Sass (Syntactically Awesome Style Sheets)}
Sass es un preprocesador de CSS, que ayuda a reducir la repetición con CSS y ahorra tiempo. Es un lenguaje de extensión CSS más estable y potente que describe el estilo de una página estructuralmente. Sus principales atributos son:
\begin{enumerate}
  \item Es un súper conjunto de CSS, lo que significa que contiene todas las características de CSS y es un preprocesador de código abierto, codificado en Ruby.
  \item Proporciona algunas características, que se utilizan para crear hojas de estilo que permiten escribir código más eficiente y fácil de mantener.
\end{enumerate}
\subsubsection{Webpack}
Webpack es un empaquetador de módulos. Webpack toma un archivo de entrada, encuentra todos los archivos de los que depende y genera un archivo que contiene todo el código de una aplicación. Con él es posible importar archivos CSS e imágenes directamente a JavaScript. Se puede compilar CoffeeScript, TypeScript, SASS y LESS. También es capaz de compilar la sintaxis de ES6 en JavaScript amigable para el navegador. En otras palabras, Webpack toma diferentes archivos (como CSS, JS, SASS, JPG, SVG, PNG, etc.) y los combina en paquetes, un paquete separado para cada tipo de archivo.


% \begin{lstlisting}[style=ES6, caption=Ejemplo de sintaxis ES5 y ES6]
%   // Extraer de objetos

%   var obj1 = { a: 1, b: 2 };
% \end{lstlisting}