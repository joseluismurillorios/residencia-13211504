Angular o React, proporcionan la interfaz de usuario reactiva de una aplicación. Utilizan componentes, son reactivos porque el usuario recibe cambios inmediatos cuando interactúa con la aplicación y, por lo general, se ejecutan dentro del navegador de un usuario (aunque ambos son \glspl{isomorfico}, capaces de ejecutarse en un servidor).
\vspace{0.8cm}

\begin{table}[H]
  \renewcommand{\arraystretch}{1.5}
  \centering
  \scriptsize
  \begin{tabular}{ |p{2cm}||p{5cm}|p{5cm}|  }
    \hline
      & Angular
      & React \\
    \hline
    Desarrollador
      & Google
      & Facebook \\
    \hline
    Definición
      & \Gls{framework}
      & Librería \\
    \hline
    Modelo de plantilla
      & HTML + Typescript
      & JSX + Javascript \\
    \hline
    Flujo
      & 2 vías
      & Unidireccional \\
    \hline
    DOM
      & Regular
      & Virtual \\
    \hline
    Lógica/Estado de la aplicación
      & Services
      & Flux/Redux \\
    \hline
  \end{tabular}
  \caption{Características de Angular.js y ReactJS}
\end{table}
\vspace{0.8cm}

Angular es un \gls{framework} con muchas herramientas integradas, para hacer solicitudes HTTP, enrutamiento y navegación, animaciones y otros. Se basa en módulos que son componentes y servicios.
\vspace{0.8cm}

React es una biblioteca de Javascript, que se puede usar para crear nuevas aplicaciones o para integrarla con una aplicación existente. React se basa en componentes pequeños y reutilizables, que administran su propio estado y luego los componen para crear interfaces de usuario complejas. Incluso si React no es tan complejo como Angular, con muchas cosas integradas, hay muchas bibliotecas que se pueden agregar para tener enrutadores (react-router) y solicitudes HTTP (\gls{axios}), manejo de estado (react-redux) entre otras más. Esto lo hace portátil y fácil de incorporar en cualquier entorno.