LAMP es un acrónimo de Linux, Apache, MySQL, PHP (Perl o Python), componentes de código abierto. Funciona como un paquete de programas que proporcionan una plataforma robusta para desarrollar e implementar aplicaciones y servidores basados en web. Durante años, ha sido la solución más efectiva para desarrollar aplicaciones web de nivel empresarial con personalización y flexibilidad mejoradas, de manera rentable. LAMP sigue siendo relevante, es muy atractivo para muchos usuarios porque es asequible y eficiente, ofreciendo una excelente alternativa a los paquetes de software comerciales. 
\subsection{Componentes}
\begin{itemize}
  \item Linux (sistema operativo)
  \item Apache (servidor web)
  \item MySQL (persistencia de datos)
  \item PHP (lenguaje de programación)
\end{itemize}

Derivados:

\begin{itemize}
  \item LAMP (con Perl o Python en lugar de PHP)
  \item LAMP (con MongoDB en lugar de MySQL)
  \item WAMP (Windows como SO)
  \item MAMP (Mac OS X como SO)
  \item XAMPP (Cualquier servidor OS + Perl o PHP + FTP)
  \item LAPP (PostgreSQL como base de datos)
\end{itemize}

\subsection{Beneficios}
Es utilizado por cientos de miles de empresas y, por lo tanto, su mantenimiento está muy bien respaldado. Con infinitos módulos, bibliotecas y complementos disponibles, puede ser adaptado a las necesidades de una empresa.\\[0.8cm]
Es posible controlar el servidor y decidir qué versiones y software se instalarán, por lo que no tiene que depender del navegador del cliente.
\subsection{Desventajas}
Debido a que es fácil de aprender, es posible caer en malas prácticas y crear aplicaciones basura. Comenzar con PHP, Python o Perl es fácil, pero dominarlo es difícil. Esto también es cierto para la seguridad en estas aplicaciones.