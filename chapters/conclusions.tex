El objetivo del proyecto era desarrollar y desplegar un sistema en la empresa Rosesland que les permitiera llevar un proceso de trabajo más automatizado. Esta compañía de venta de arreglos florales, como muchas organizaciones, necesitaba migrar sistemas operativos clave a plataformas web para aprovechar beneficios como la ubicuidad del acceso web y la arquitectura orientada a servicios. Para lograr esto, se desarrolló un servidor basado en Node.js que se conecta con los servicios de venta en linea de Shopify, así como una aplicación web creada con React.js siguiendo la metodología del diseño atómico e implementando Redux para el manejo del estado, el sistema utiliza los servicios de Google Firebase para la autenticación y el almacenamiento de los datos.
\vspace{0.8cm}

Los servicios de venta en linea son cada vez mas populares tanto en tiendas minoristas como en mayoristas,  este sistema se enfocó en mejorar la situación laboral de los empleados y ofrecer un servicio mas agradable a los clientes. Se canalizaron principios de usabilidad en una web de ventas para crear una plataforma que influye positivamente en el uso de el sistema como herramienta para los trabajadores. Una vez finalizadas las pruebas y haber garantizado el funcionamiento de todos los módulos se puede concluir que se obtuvieron los resultados deseados. El sistema tuvo un gran desempeño y todas las actividades de la empresa fueron mas fáciles de realizar. Uno de los principales beneficios que aporta se ve reflejado en la capacidad para analizar y administrar las ventas como antes no se había imaginado.
\vspace{0.8cm}

Conforme el proyecto evolucionaba en su proceso de desarrollo, pruebas y ejecución, se adquirieron conocimientos importantes. Los procedimientos establecidos al crear este proyecto sirven como base para elaborar otras aplicaciones, su código es escalable y permite integrar un gran numero de servicios y \glspl{framework}, por lo que su limitación solo estará marcada por las necesidades de los usuarios.
\vspace{0.8cm}

