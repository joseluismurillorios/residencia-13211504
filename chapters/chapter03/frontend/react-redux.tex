Redux practica la teoría de flujo de datos unidireccional y se convirtió en un patrón de facto como tecnología de gestión de estado para aplicaciones ReactJS. React utiliza \textit{props} (abreviatura de propiedades) en un componente que permite el uso de variables no estáticas. Con la ayuda de \textit{props}, podemos pasar estas variables a varios otros componentes (secundarios) desde el componente principal.
\vspace{0.8cm}

La conexión del Store de Redux con los componentes de React, es mediante un componente llamado \code{Provider} del módulo \code{react-redux}. En React para compartir datos entre componentes, un estado tiene que vivir en el componente principal. Este componente principal proporciona un método para actualizar este estado y se pasa como \textit{props} a estos componentes. El único propósito de \code{Provider} es agregar el Store al contexto del componente de la Aplicación, para que todos los componentes secundarios puedan acceder a ella mediante la función \code{connect} de \code{react-redux}. \code{Provider} envuelve a la aplicación React y hace que sea consciente de el Store.
\vspace{0.8cm}

\lstinputlisting[style=ES6, caption=Fragmento de código de la aplicación React principal]{code/redux-react.js}

La función \code{connect} ayuda a conectar un componente con el estado de la aplicación, se debe definir una función especial llamada \code{mapStateToProps} para describir que partes del estado actual de Redux se desean pasar al componente que está envolviendo y una función de nombre \code{mapDispatchToProps} conecta las acciones de Redux con React \textit{props}. De esta manera, un componente React conectado podrá enviar mensajes a el Store.
\vspace{0.8cm}

\lstinputlisting[style=ES6, caption=Ejemplo de conexión de un componente React con el estado Redux]{code/connect.js}
% \begin{enumerate}
%   \item El primer argumento es
% \end{enumerate}