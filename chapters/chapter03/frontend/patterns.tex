Las aplicaciones web modernas tienen una estructura de programa compleja, debido a las funcionalidades que proporcionan en sus interfaces de usuario. Escribir manualmente un código de programa puede dar como resultado una calidad y contenido desiguales en partes individuales de la aplicación. Mantener tales aplicaciones desarrolladas es más difícil. Debido a esto, las aplicaciones web a menudo se desarrollan utilizando diferentes frameworks y patrones de diseño.
\vspace{0.8cm}

Los patrones de diseño facilitan la reutilización de diseños y arquitecturas exitosas. Los patrones de diseño ayudan a elegir alternativas de diseño que hacen que un sistema sea reutilizable y evitar alternativas que comprometan la reutilización. Pueden incluso mejorar la documentación y el mantenimiento de los sistemas existentes.
\vspace{0.8cm}

Los patrones de diseño pueden ser increíblemente útiles si se usan en las situaciones correctas y por las razones correctas. Cuando se usan estratégicamente, pueden hacer que un programador sea significativamente más eficiente al permitirle el usar métodos refinados por otros y evitar `reinventar la rueda'. También proporcionan un lenguaje común útil para conceptualizar problemas y soluciones repetidos cuando se discute con otros o se maneja el código en equipos más grandes.