El desarrollo web es un campo en constante cambio: la forma en que construimos sitios web hoy en día es completamente diferente de cómo solíamos hacerlo hace un par de años. Con la gran cantidad de herramientas disponibles y las nuevas que aparecen todos los días, la mayoría de las veces los desarrolladores se sienten confundidos sobre qué camino tomar.
\vspace{0.8cm}

Las aplicaciones web modernas tienen una estructura de programa compleja, debido a las funcionalidades que proporcionan en sus interfaces de usuario. Escribir manualmente un código de programa puede dar como resultado una calidad y contenido desiguales en partes individuales de la aplicación. Mantener tales aplicaciones desarrolladas es más difícil. Debido a esto, las aplicaciones web a menudo se desarrollan utilizando diferentes frameworks y patrones de diseño.

\subsection{Patrones de diseño}
Los patrones de diseño facilitan la reutilización de diseños y arquitecturas exitosas. Los patrones de diseño ayudan a elegir alternativas de diseño que hacen que un sistema sea reutilizable y evitar alternativas que comprometan la reutilización. Pueden incluso mejorar la documentación y el mantenimiento de los sistemas existentes.
\vspace{0.8cm}

Los patrones de diseño pueden ser increíblemente útiles si se usan en las situaciones correctas y por las razones correctas. Cuando se usan estratégicamente, pueden hacer que un programador sea significativamente más eficiente al permitirle el usar métodos refinados por otros y evitar `reinventar la rueda'. También proporcionan un lenguaje común útil para conceptualizar problemas y soluciones repetidos cuando se discute con otros o se maneja el código en equipos más grandes.

\subsection{Redux}
Redux es un administrador de estado predecible para aplicaciones JavaScript basado en el patrón de diseño Flux. A medida que una aplicación crece, se hace difícil mantenerla organizada y mantener el flujo de datos. Redux resuelve este problema administrando el estado de la aplicación con un único objeto global llamado Redux Store. Los principios fundamentales de Redux ayudan a mantener la coherencia en toda la aplicación, lo que facilita la depuración y las pruebas. Redux se puede conectar con cualquier biblioteca de JavaScript. Sin embargo, funciona muy bien con ReactJS debido a su naturaleza funcional.
\vspace{0.8cm}

Redux ayuda a separar el estado de la aplicación, crea un almacén global que reside en el nivel superior de una aplicación y alimenta con el estado a todos los componentes internos. A diferencia de Flux, Redux no tiene múltiples objetos de almacenamiento. El estado completo de la aplicación está dentro de un objeto, y potencialmente podría intercambiar la capa de vista con otra biblioteca con el almacenamiento intacto.

\subsubsection{Redux/Flux}
Redux adoptó un gran número de restricciones de la arquitectura Flux: las acciones encapsulan la información para que el Redux Reducer actualice el estado de manera determinista, el estado es un Redux Store \gls{singleton}. El despachador de Flux único se reemplaza con múltiples Redux Reducers pequeños que recogen información de las acciones y la `reducen' a un nuevo estado que luego se guarda en el Redux Store. Cuando se cambia el estado en el Store, la Vista según la suscripción recibe propiedades llamadas props.
\vspace{0.8cm}

\begin{figure}[H]
  \centering
  \includegraphics[width=0.9\textwidth]{flux-redux}
  \caption{Comparación Flux/Redux. (Fuente: Elaboración propia)}
\end{figure}
\vspace{0.8cm}

Para trabajar con Redux se necesitan tres cosas:
\begin{itemize}
  \item Actions (acciones): estos son objetos que deben tener dos propiedades, una que describe el tipo de acción y otra que describe lo que se debe cambiar en el estado de la aplicación.

  \item Reducers (reductores): son funciones que implementan el comportamiento de las acciones. Cambian el estado de la aplicación, en función de la descripción de la acción y la descripción del cambio de estado.

  \item Store (almacén): reúne las acciones y los reducers, manteniendo y cambiando el estado de toda la aplicación; solo hay una store.
\end{itemize}

\subsubsection{Redux Store}
Redux Store contiene un objeto del estado global de la aplicación. Esta actualiza el estado y notifica los componentes suscritos.
\vspace{0.8cm}

\lstinputlisting[style=ES6, caption=Fragmento de código para inicializar el Redux Store]{code/redux-store.js}

\subsubsection{Redux Reducer}
Un Redux Reducer es solo una función pura de JavaScript. Recibe dos parámetros: el estado actual y la acción. Una función pura es aquella que devuelve exactamente la misma salida para la entrada dada. El estado es el objeto Redux Store completo, la acción es el objeto despachado con un tipo requerido y un payload opcional.
\vspace{0.8cm}

\lstinputlisting[style=ES6, caption=Fragmento de código del reducer común de la app]{code/redux-reducer.js}

\subsubsection{Acciones Redux}
La única forma de cambiar el estado es enviando una señal al Store. Esta señal es una acción. Entonces "despachar una acción" significa enviar una señal a Redux Store.
\vspace{0.8cm}

\lstinputlisting[style=ES6, caption=Fragmento de una acción que actualiza el estado]{code/redux-action.js}

Este es un modelo conveniente y directo para estructurar datos en una aplicación y presentarlos en el cliente. La aplicación tiene un estado raíz. Un cambio de estado desencadena actualizaciones de vista. Solo las funciones especiales pueden modificar el estado. Una interacción del usuario activa estas funciones especiales de cambio de estado. Solo se produce un cambio a la vez. Esto significa que el estado central no puede desencadenar ninguna otra acción. Solo una entrada del usuario puede desencadenar otra acción \cite{mukhiya}.

\subsection{Integración React/Redux}
Redux practica la teoría de flujo de datos unidireccional y se convirtió en un patrón de facto como tecnología de gestión de estado para aplicaciones ReactJS. React utiliza \textit{props} (abreviatura de propiedades) en un componente que permite el uso de variables no estáticas. Con la ayuda de \textit{props}, podemos pasar estas variables a varios otros componentes (componentes secundarios) desde el componente principal.
\vspace{0.8cm}

La conexión del Store de Redux con los componentes de React, es mediante un componente llamado Provider. En React para compartir datos entre componentes, un estado tiene que vivir en el componente principal. Este componente principal proporciona un método para actualizar este estado y se pasa como \textit{props} a estos componentes. El único propósito de Provider es agregar el Store al contexto del componente de la Aplicación, para que todos los componentes secundarios puedan acceder a ella. Provider envuelve a la aplicación React y hace que sea consciente de el Store.
\vspace{0.8cm}

\lstinputlisting[style=ES6, caption=Fragmento de código de la aplicación React principal]{code/redux-react.js}


\subsection{Diseño Atómico (Atomic Design)}
El diseño atómico es una metodología compuesta por cinco etapas distintas que trabajan juntas para crear sistemas de diseño de interfaz de una manera más deliberada y jerárquica. Las cinco etapas del diseño atómico son:
\vspace{0.8cm}

\begin{enumerate}
  \item Átomos
  \item Moléculas
  \item Organismos
  \item Plantillas
  \item Páginas
\end{enumerate}
\vspace{0.8cm}

El diseño atómico no es un proceso lineal, sino un modelo mental que nos ayuda a pensar en nuestras interfaces de usuario como un todo cohesivo y una colección de partes al mismo tiempo. Cada una de las cinco etapas juega un papel clave en la jerarquía de nuestros sistemas de diseño de interfaz \cite{frost}.
\vspace{0.8cm}


\begin{figure}[H]
  \centering
  \includegraphics[width=1\textwidth]{atomic}
  \caption{Metodología de diseño atómico.}
\end{figure}

\subsection{Estructura de la aplicación web}
Rosesland cuenta con dispositivos tipo tablet, por lo que el diseño de la interfaz de usuario debe aprovechar las ventajas que ofrece este entorno interactivo. Se pretende resumir el proceso de venta, elaboración y entrega de productos en dos secciones. También se agrega una sección de administración de usuarios y una página con información general de la florería.
\vspace{0.8cm}

La interfaz de usuario juega un papel muy importante en el aumento de la usabilidad de una aplicación, dado que la IU ofrece al usuario una vista abstracta de todo el sistema, el éxito del sistema depende en gran medida de ello. Por lo tanto, el diseño de la interfaz de usuario debe tener la importancia adecuada en el proceso del ciclo de vida del diseño del sistema.
\vspace{0.8cm}

% \begin{figure}[H]
%   \centering
%   \includegraphics[width=0.75\textwidth]{secuence}
%   \caption{Diagrama de secuencia del proceso de autorización.}
% \end{figure}
% \vspace{0.8cm}
\subsubsection{Módulo de inicio de sesión}
Uno de los desafíos es cómo implementar un esquema de autenticación y autorización flexible, seguro y eficiente. Parece confuso diferenciar entre autenticación y autorización. De hecho, es muy simple.

\begin{itemize}
  \item Autenticación: se refiere a verificar `quién es usted', por lo que debe usar el nombre de usuario y la contraseña para la autenticación.

  \item Autorización: se refiere a lo que `puede hacer', por ejemplo, acceder, editar o eliminar datos a algunos documentos, esto sucede después de la verificación.
\end{itemize}

Firebase Authentication proporciona servicios de back-end, SDK fáciles de usar y bibliotecas de interfaz de usuario para autenticar a los usuarios en la aplicación. En el código ejemplo \ref{login} se muestran los métodos necesarios para el manejo de sesiones del proyecto.
\vspace{0.8cm}

\lstinputlisting[style=ES6, label=login, caption=Fragmento de código del manejo de sesión de usuario]{code/login.js}

Gracias a la simplicidad y efectividad de los servicios de Firebase, este proceso es utilizado por muchas aplicaciones y servicios web en la actualidad.
\vspace{0.8cm}

\begin{figure}[H]
  \centering
  \includegraphics[width=1\textwidth]{app-login}
  \caption{Diseño final de la sección de inicio de sesión.}
\end{figure}

\subsubsection{Interfaz de usuario principal}
Después del inicio de sesión exitoso, el usuario es dirigido a la página principal que proporciona enlaces a los módulos del sistema. Esta sección contiene los componentes visuales de cada paso en el proceso de venta, manufactura y entrega, aquí se muestran todas las ordenes del día creadas mediante la conexión con Shopify, así como las creadas por la aplicación.

Este módulo permite a los vendedores asignar los floristas que elaboraran los pedidos y al administrador de logística definir los chóferes que repartirán la entrega. Las ordenes mostradas se filtran para mostrar solo la información necesaria a cada usuario en específico.
\vspace{0.8cm}

\begin{figure}[H]
  \centering
  \includegraphics[width=1\textwidth]{app-main}
  \caption{Diseño final de la sección principal.}
  \label{main-ui}
\end{figure}

En la figura \ref{main-ui} se muestran los componentes principales de la interfaz de usuario (vista por el administrador). Por el lado izquierdo y en la parte superior se tienen accesos a los módulos del sistema y al perfil del usuario, El cuerpo contiene un resumen de las ventas (que solo es visible por el administrador), una tabla con la información de las ordenes del día y la opción de ver ordenes creadas para días posteriores. Las ordenes sin revisar se marcan en un tono azul y el estado de cada orden se identifica con un color:

\begin{itemize}
  \item Gris: orden pendiente.
  \item Rojo: orden iniciada.
  \item Amarillo: orden en proceso de entrega.
  \item Verde: orden finalizada.
\end{itemize}

El diseño responsivo permite al administrador llevar control del estatus de las ordenes en todas las etapas del proceso desde un dispositivo celular.
\vspace{0.8cm}

\begin{figure}[H]
  \centering
  \includegraphics[width=0.4\textwidth]{app-mobile}
  \caption{Diseño de la aplicación móvil.}
\end{figure}
\vspace{0.8cm}

\subsubsection{Interfaz de usuario de la sección de ventas}
Para crear ordenes dentro del establecimiento se accede a esta sección. Aquí se seleccionan los productos, la dirección de entrega, la dedicatoria y la información del emisor y remitente de la orden. Si se llega a cometer un error con la orden una vez creada, este mismo módulo permite editarla.
\vspace{0.8cm}

\begin{figure}[H]
  \centering
  \includegraphics[width=1\textwidth]{app-sales}
  \caption{Diseño final de la sección de ventas.}
\end{figure}

El componente de catalogo de productos (figura \ref{catalog} y \ref{single-product}) se muestra en una ventana emergente que puede ser deslizada horizontalmente con el tacto, con la ayuda del mouse o haciendo clic en sus elementos. Un producto puede tener variantes de color de flor, color de papel, cantidad de flores, etc. Cada producto puede también contener extras y una dedicatoria (figura \ref{single-product}).

\begin{figure}[H]
  \centering
  \includegraphics[width=0.6\textwidth]{app-catalog}
  \caption{Componente de catálogo de productos.}
  \label{catalog}
\end{figure}

\begin{figure}[H]
  \centering
  \includegraphics[width=0.6\textwidth]{app-single}
  \caption{Componente de producto.}
  \label{single-product}
\end{figure}

El componente que se utiliza para seleccionar la ubicación de destino utiliza los servicios de Google Maps. Este componente que se abre en una ventana emergente permite seleccionar un punto en un mapa, este punto esta restringido por una capa que delimita a la ciudad de Tijuana.
\vspace{0.8cm}

\begin{figure}[H]
  \centering
  \includegraphics[width=0.8\textwidth]{app-map}
  \caption{Componente de Google Map.}
\end{figure}

Una vez seleccionada la ubicación, el sistema ejecuta una función que recibe las coordenadas y calcula la distancia entre el punto elegido y la ubicación de la florería. Para tomar en cuenta la curvatura de la tierra, esta función utiliza la fórmula del semiverseno (haversine), la cual es una ecuación importante en la navegación, proporciona distancias de entre dos puntos en una esfera desde sus longitudes y latitudes \cite{anisya}.

\[ 
  haversin\left(\frac{d}{R}\right) = haversin(\Delta\varphi) + cos(\Delta\varphi)haversin(\Delta\lambda)
\]

\textbf{Donde:}\\
\-\hspace{0.5cm} haversin es la función haversine, haversin($\Theta$) = $sen^2$ ($\Theta$/2) = (1-cos ($\Theta$))/2\\
\-\hspace{0.5cm} d es la distancia entre dos puntos\\
\-\hspace{0.5cm} R es el radio de la esfera\\
\-\hspace{0.5cm} $\Delta\varphi$ es la diferencia de latitudes\\
\-\hspace{0.5cm} $\Delta\lambda$ es la diferencia de longitudes
\vspace{0.8cm}

\lstinputlisting[style=ES6, label=login, caption=Función para calcular distancia entre dos puntos geográficos]{code/harvesine.js}

\subsubsection{Interfaz para el área de manufactura y entrega}
En esta sección se muestra el detalle de las ordenes, su principal función es desplegar en pantalla la información necesaria para elaborar y entregar los pedidos. Esta parte de la aplicación es visible para todos los usuarios, pero la información mostrada es distinta para cada empleado. Los floristas y chóferes solo pueden ver las ordenes asignadas a su usuario, los administradores y empleados de venta tienen acceso a todas las ordenes, debido a que en esta sección se encuentran los enlaces para editar las ordenes e imprimir los recibos en caso de ser solicitados.
\vspace{0.8cm}

\begin{figure}[H]
  \centering
  \includegraphics[width=1\textwidth]{app-orders}
  \caption{Diseño final de la sección de manufactura y entrega.}
  \label{production-ui}
\end{figure}

En la figura \ref{production-ui} se observan dos módulos principales; en el lado izquierdo se tiene una lista con las ordenes del día organizadas por hora de entrega, además de opciones para ver todas las fechas y/o filtrarlas por estatus. El lado derecho consta de un mapa con la dirección de entrega (invisible para los floristas o en caso de que la orden sea para recoger en la florería) y un tablero con las especificaciones de la orden incluidas las imágenes de los arreglos para proporcionar una ayuda visual en el área de elaboración de productos.
\vspace{0.8cm}

\subsubsection{Interfaz de administración de usuarios}
Este panel permita agregar y editar la información de los usuarios incluidos sus permisos, solo los administradores tienen acceso a este módulo; a diferencia del módulo de inicio de sesión, las actividades realizadas en estos componentes deben ser validadas por Firebase Authentication en el lado del servidor.
\vspace{0.8cm}

\begin{figure}[H]
  \centering
  \includegraphics[width=1\textwidth]{app-admin}
  \caption{Diseño final de la sección de administración de usuarios.}
  \label{admin-ui}
\end{figure}

Por el lado izquierdo se tiene una lista con los usuarios con un indicador del estatus de su conexión en el sistema; en color rojo se muestran los empleados inactivos y en verde los empleados conectados al sistema. En el lado derecho se encuentran los campos con la información del usuario y la opción de actualizarlo o eliminarlo. En esta sección se aprecia con todo detalle los beneficios de utilizar Redux y el Diseño Atómico, distintos componentes conectados con un mismo estado (figura \ref{admin-atomic}).
\vspace{0.8cm}

\begin{figure}[H]
  \centering
  \includegraphics[width=1\textwidth]{app-atomic}
  \caption{Implementación Redux y Diseño Atómico.}
  \label{admin-atomic}
\end{figure}

\subsubsection{Evaluación de la interfaz de usuario}
El objetivo del proyecto es señalar los problemas y desafíos que surgen del uso de una pantalla táctil como dispositivo de entrada en una aplicación web. Para lograr esto, se desarrolló un prototipo, basado en pautas teóricas. El prototipo fue evaluado en sesiones informales donde los usuarios lo probaron y hablaron sobre cómo lo percibieron. Se alentó a los sujetos a sugerir soluciones alternativas y criticar las soluciones que encontraron negativas.
\vspace{0.8cm}

El prototipo inicial sufrió muy pocos cambios, siendo principalmente por motivos de funcionalidad y no estéticos. Todos los empleados que sometieron a prueba la aplicación acordaron que el diseño era fácil de entender y se podían acostumbrar a el rápidamente. El resultado de la evaluación demostró que la interfaz responsiva en una pantalla táctil genera una ayuda visual importante en todas las áreas de la empresa.