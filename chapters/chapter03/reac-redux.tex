Para conectar el Store de Redux con React, es mediante un componente llamado Provider. El único propósito de Provider es agregar el Store al contexto del componente de la Aplicación, para que todos los componentes secundarios puedan acceder a ella. Provider envuelve a la aplicación React y hace que sea consciente de el Store. \\[0.8cm]
\lstinputlisting[style=ES6, caption=Fragmento de código de la aplicación React principal]{code/redux-react.js}
