Los patrones de diseño facilitan la reutilización de diseños y arquitecturas exitosas. Los patrones de diseño ayudan a elegir alternativas de diseño que hacen que un sistema sea reutilizable y evitar alternativas que comprometan la reutilización. Pueden incluso mejorar la documentación y el mantenimiento de los sistemas existentes.
\subsection{Redux}
Redux es un administrador de estado predecible para aplicaciones JavaScript basado en el patrón de diseño Flux. A medida que una aplicación crece, se hace difícil mantenerla organizada y mantener el flujo de datos. Redux resuelve este problema administrando el estado de la aplicación con un único objeto global llamado Store. Los principios fundamentales de Redux ayudan a mantener la coherencia en toda la aplicación, lo que facilita la depuración y las pruebas.
\subsubsection{Redux Store}
Redux Store contiene un objeto del estado global de la aplicación. Esta actualiza el estado y notifica los componentes suscritos. \\[0.8cm]
\lstinputlisting[style=ES6, caption=Fragmento de código para inicializar el Store]{code/redux-store.js}

\subsubsection{Redux Reducer}
Un Reducer es solo una función pura de JavaScript. Recibe dos parámetros: el estado actual y la acción. Una función pura es aquella que devuelve exactamente la misma salida para la entrada dada. El estado es el objeto Store completo, la acción es el objeto despachado con un tipo requerido y un payload opcional. \\[0.8cm]
\lstinputlisting[style=ES6, caption=Fragmento de código del reducer común de la app]{code/redux-reducer.js}

\subsubsection{Acciones Redux}
La única forma de cambiar el estado es enviando una señal a el Store. Esta señal es una acción. Entonces "despachar una acción" significa enviar una señal a el Store. \\[0.8cm]
\lstinputlisting[style=ES6, caption=Fragmento de código de la acción que valida a un administrador]{code/redux-action.js}
