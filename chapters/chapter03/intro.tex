Las microempresas constituyen una potencia impulsora y se consideran entre las principales fuentes de trabajo, no obstante, es muy común encontrar microempresas con dificultades de crecimiento, principalmente debido a una mala administración o por falta de aprovechamiento de sus recursos.
\vspace{0.8cm}

A continuación se describen los procedimientos necesarios para elaborar un sistema que permita a la empresa Rosesland llevar control de las ventas, elaboración y entrega de productos, así como incorporar los servicios de su tienda en linea al sistema. Permitiendo a la empresa tomar ventaja de las tecnologías web mas recientes, aprovechando al máximo los recursos existentes de la compañía.

\section{E-Commerce (Comercio electrónico)}
El comercio electrónico se refiere al proceso de compra o venta de productos o servicios a través de Internet. Las compras en línea se están volviendo cada vez más populares debido a la velocidad y facilidad de uso para los clientes. Las actividades de comercio electrónico, como la venta en línea, pueden dirigirse a consumidores u otras empresas. Vender en línea puede ayudar a su empresa a llegar a nuevos mercados y aumentar sus ventas e ingresos (ya sea a través de su propio sitio web o de un sitio de mercado electrónico).

\begin{figure}[H]
  \centering
  \includegraphics[width=0.9\textwidth]{rosesland}
  \caption{Tienda en linea actual de la empresa.}
\end{figure}

\subsubsection{Shopify}
Shopify es un servicio web que le permite configurar una tienda en línea para vender sus productos. 
Le da la facilidad organizar sus productos, personalizar el diseño de su tienda, 
aceptar pagos con tarjeta de crédito, rastrear y responder a pedidos. 
Shopify.com permite a los vendedores elegir entre opciones de diseño gratuitas 
o diseños personalizados creados por los usuarios.

\begin{figure}[H]
  \centering
  \includegraphics[width=0.9\textwidth]{shopify}
  \caption{Interfaz del administrador Shopify.}
\end{figure}

\newpage
\section{Planeación del sistema}
Los procesos internos de la empresa se llevan a cabo realizando comandas en papel que los trabajadores llenan manualmente; esto representa un grave problema ya que los clientes tienen una mala experiencia de compra y los empleados tienden a confundir los pedidos. Además no se lleva un registro digital de las ventas, el cual es importante para analizar el estatus de las ordenes. 
\vspace{0.8cm}

\begin{figure}[H]
  \centering
  \includegraphics[width=1\textwidth]{use-case}
  \caption{Diagrama de casos de uso del sistema. (Fuente: Elaboración propia)}
\end{figure}
\vspace{0.8cm}

\section{Requerimientos del proyecto}
El producto final está enfocado en incrementar las ventas de la empresa Rosesland, con la ayuda de una plataforma que administre y mejore sus procesos internos. La finalidad del proyecto es disminuir los tiempos de ventas, elaboración y entrega de productos, así como la de crear un vínculo entre el servidor y su infraestructura actual extendiendo sus capacidades.
\vspace{0.8cm}

El sistema debe adaptarse al sistema operativo Linux. Anticipado a futuros cambios en la plataforma se decide trabajar con Node.js, ya que es de código abierto y multiplataforma, esto permite beneficiarse de la reutilización del código y la falta de cambio de contexto. Las aplicaciones Node.js están escritas en JavaScript puro y pueden ejecutarse dentro del entorno Node.js en Windows, Linux, etc.
\vspace{0.8cm}

La interfaz gráfica de usuario de la aplicación debe ser responsiva y funcionar en la mayoría de los navegadores web modernos, además debe ser fácil de aprender, idealmente requerir poco entrenamiento.

\subsection{Herramientas}
Node.js permite incorporar herramientas poderosas a proyectos de cualquier tipo. Esto incluye todo, desde bibliotecas y frameworks como jQuery y AngularJS hasta procesadores de código como Webpack. Los paquetes vendrán en una carpeta típicamente llamada node\_modules, que también contendrá un archivo package.json. Este archivo contiene información sobre todos los paquetes, incluidas las dependencias, que son módulos adicionales necesarios para usar un paquete en particular.

\subsubsection{Dependencias de desarrollo}
Las dependencias de desarrollo son aquellas que se utilizan dentro del entorno de programación. Aquí se incluyen herramientas que no forman parte del ejecutable final como lo son lo preprocesadores, empaquetadores y analistas de código. Los principales módulos de desarrollo son los siguientes:
\begin{itemize}
  \item Babel: es un compilador de JavaScript utilizado para convertir código ECMAScript 2015+ (ES6+) en una versión que pueda ser ejecutado en motores JavaScript más antiguos.
  \item Webpack: es un empaquetador de módulos principalmente para JavaScript, pero puede transformar archivos front-end como HTML, CSS e imágenes si se incluyen los complementos correspondientes.
  \item ESLint: es una herramienta de análisis de código estático para identificar patrones problemáticos encontrados en el código JavaScript.
  \item PostCSS: es una herramienta de desarrollo de software que utiliza complementos basados en JavaScript para automatizar las operaciones de rutina de CSS. Con este modulo es posible analizar CSS, agregar prefijos de proveedor a las reglas CSS, ejecutar optimizaciones enfocadas, para garantizar que el resultado final sea lo más pequeño posible para un entorno de producción.
  \item Pug: es un preprocesador que simplifica la tarea de escribir HTML. También agrega funcionalidades, como objetos Javascript, condiciones, bucles, mixins y plantillas.
\end{itemize}

\subsubsection{Estructura del proyecto}
La estructura del proyecto Node.js está influenciada por preferencias personales, la arquitectura del proyecto y la estrategia de inyección de módulos que se está utilizando. Para tener una estructura FERN, es imperativo separar el código fuente del servidor y el utilizado por el cliente, ya que el código del lado del cliente o front-end probablemente se minimizará y se enviará al navegador y es público en su naturaleza básica. Y el lado del servidor o el back-end proporcionarán API para realizar operaciones CRUD.

\subsubsection{Directorios}
\begin{itemize}
  \item node\_modules: Directorio oculto que contiene las dependencias generales del proyecto, este archivo debe ser ignorado por el controlador de versiones.
  \item server: Aquí se encuentran los archivos requeridos por el servidor para el enrutamiento, la conexión con la base de datos y las funciones de utilidad del back-end.
  \item src: Este directorio concentra todos los elementos necesarios para producir nuestra aplicación cliente.
  \item views: En esta carpeta se incluyen los templates necesarios para generar las vistas HTML de nuestro proyecto.
  \item www: Su propósito es contener los archivos del cliente, aquí podemos encontrar el código de producción de nuestra aplicación web compilado y minificado, así como las hojas de estilo, imágenes, fuentes tipográficas, vectores, etc.
\end{itemize}

\subsubsection{Archivos principales}
\begin{itemize}
  \item .babelrc: Contiene los mecanismos necesarios para compilar sintaxis moderna de JavaScript y una lista con los navegadores web a los que se requiere dar enfoque.
  \item .env: Documento oculto que contiene las variables del entorno, este archivo es ignorado por el controlador de versiones.
  \item .eslintrc.js: En este archivo se declaran las reglas para identificar e informar sobre patrones o errores encontrados en el código.
  \item .gitignore: Aquí se describen los archivos y directorios que deben ser ignorados por Git.
  \item index.js: Entrada principal de nuestro servidor, contiene el código necesario para que el proyecto funcione correctamente.
  \item package.json: Este archivo puede contener muchos metadatos sobre su proyecto. Pero principalmente se usa para dos cosas:
  \begin{itemize}
    \item Gestionar dependencias de su proyecto
    \item Scripts, que ayudan a generar compilaciones, ejecutar pruebas y otras cosas con respecto al proyecto
  \end{itemize}
\end{itemize}

\begin{figure}[H]
  \centering
  \includegraphics[width=0.3\textwidth]{app}
  \caption{Vista general de la raíz del proyecto.}
\end{figure}

\newpage
\subsubsection{Variables del entorno}
El acceso a las variables de entorno en Node.js es compatible desde el primer momento. Cuando el proceso Node.js se inicia, automáticamente proporciona acceso a todas las variables de entorno existentes al crear un objeto env como propiedad del objeto global del proceso.
\vspace{0.8cm}

\lstinputlisting[style=ES6, caption=Variables principales del sistema]{code/env.txt}


\newpage
\section{Configuración del servidor (Backend)}
Si alguna vez ha utilizado PHP o ASP, probablemente esté acostumbrado a la idea de que el servidor web (Apache o IIS, por ejemplo) sirve sus archivos estáticos para que un navegador puede verlos a través de la red. Node ofrece un paradigma diferente al de un servidor web tradicional: la aplicación es el servidor web. Node simplemente proporciona las bases para que se pueda construir un servidor web. 

\subsection{Servidor NodeJS}
El modelo de E/S impulsado por eventos sin bloqueo le brinda a NodeJS un rendimiento muy atractivo, superando fácilmente los entornos de servidores como PHP y Ruby on Rails, que bloquean las E/S y manejan múltiples usuarios simultáneos en hilos separados para cada uno. Algo importante que se debe saber es que NodeJS no es un \gls{framework} sino un entorno, hay \glspl{framework} que funcionan con Node, como Express y Sails, lo que facilita la creación de aplicaciones.
\vspace{0.8cm}

\begin{figure}[H]
  \centering
  \includegraphics[width=0.8\textwidth]{node-traditional}
  \caption{Comparación de node y servidores tradicionales.}
\end{figure}

Un servidor Node.js tiene un solo subproceso de bucle de eventos (event-loop) que espera E/S en sockets y archivos. Una vez que los datos están listos, activa el método de evento correspondiente y espera hasta que regrese antes de esperar nuevamente por más eventos de E/S. Dado que todas las operaciones de E/S no bloquean, se asegurará de que todo se ejecute correctamente tan pronto como la entrada esté disponible sin ningún bloqueo y sin que se tenga lidiar con problemas de subprocesos múltiples.

\newpage
\subsubsection{Programación basada en eventos}
La filosofía central detrás de NodeJS es la programación basada en eventos. Significa que, el programador,  debe comprender qué eventos están disponibles y cómo responder a ellos. Muchas personas se introducen en la programación basada en eventos mediante la implementación de una interfaz de usuario: el usuario hace clic en algo y se dispara el `evento clic'. Es una buena metáfora, porque se entiende que el programador no tiene control sobre cuándo, o si el usuario va a hacer clic en algo, por lo que la programación basada en eventos es realmente bastante intuitiva \cite{ethan}.
\vspace{0.8cm}

\lstinputlisting[label={node-server}, style=ES6, caption=Configuración servidor NodeJS básico]{code/node-server.js}
En el ejemplo de código \ref{node-server}, el evento es implícito: el evento que se está manejando es una solicitud HTTP. El método http.createServer toma una función como argumento; esta función se invocará cada vez que se realice una solicitud HTTP. El programa simplemente establece el tipo de contenido en texto sin formato y envía la cadena `Hola, mundo!'.


\subsection{Configuración Express}
Express.js es un `marco de aplicación web NodeJS minimalista y flexible' \cite{express}. Es una capa delgada de características, fundamental para cualquier aplicación web, agrega tres características poderosas: enrutamiento, mejores manejadores de solicitudes y vistas.

\subsubsection{Enrutamiento}
Enrutamiento se refiere al mecanismo para servir al cliente el contenido que ha solicitado. Para las aplicaciones cliente/servidor basadas en web, el cliente especifica el contenido deseado en la URL (ruta y cadena de consulta).\\[0.8cm]
Cuando una aplicación Express.js se está ejecutando, escucha las solicitudes. Cada solicitud entrante se procesa de acuerdo con una cadena definida de middlewares y rutas que comienzan de arriba a abajo. Este aspecto es importante porque le permite controlar el flujo de ejecución \cite{azat}.
\vspace{0.8cm}

\begin{figure}[H]
  \centering
  \includegraphics[width=0.8\textwidth]{express-router}
  \caption{Cada función de middleware en la pila se ejecuta antes que las que están debajo de ella.}
\end{figure}

\newpage
\lstinputlisting[style=ES6, caption=Fragmento de la configuración de enrutamiento del sistema]{code/express-router.js}

\subsection{Webhooks}
Un webhook permite que servicios de terceros envíen actualizaciones en tiempo real a su aplicación. Las actualizaciones se activan por algún evento o acción por parte del proveedor de webhook, y se envían a su aplicación a través de solicitudes HTTP. Cuando recibe la solicitud, la maneja con una lógica personalizada, como enviar un correo electrónico o almacenar los datos en una base de datos.

\subsection{Webhooks Shopify}
Un webhook se puede usar para recibir notificaciones sobre eventos particulares en una tienda en linea. Después de suscribirse a un webhook, puede permitir que su aplicación ejecute código inmediatamente después de que ocurran eventos específicos en las tiendas que tienen su aplicación instalada, en lugar de tener que hacer llamadas API periódicamente para verificar su estado. Por ejemplo, puede configurar un webhook para activar una acción en su aplicación cuando un cliente crea un carrito de compras o cuando un comerciante cree un nuevo producto en su administrador de Shopify. Al usar las suscripciones de webhooks, puede hacer menos llamadas API en general, lo que garantiza que la aplicación sea más eficiente y se actualice rápidamente \cite{webhook}.
\vspace{0.8cm}

\begin{figure}[H]
  \centering
  \includegraphics[width=1\textwidth]{webhook}
  \caption{Panel para crear webhooks en Shopify.}
\end{figure}

Después de configurar una suscripción de webhook, los eventos que especificó activarán una notificación de webhook cada vez que ocurran. Esta notificación contiene información JSON y encabezados HTTP que proporcionan contexto.

\newpage
\subsection{Configuración de webhook en Express.js}
En lugar de extraer información a través del API, los webhooks enviarán información a su punto final.
\vspace{0.8cm}

\lstinputlisting[style=ES6, caption=Ruta que capta el webhook lanzado desde Shopify]{code/webhook.js}

\subsection{Websockets con Socket.IO}
Si bien la base de datos Firebase Firestore proporciona una capa de conexión en tiempo real, es necesario introducir un nuevo método de comunicación permanente para notificar de eventos como lo son la presencia de usuarios activos y la visualización de las ordenes. Socket.IO permite la comunicación bidireccional entre el cliente y el servidor. Las comunicaciones bidireccionales se habilitan cuando un cliente tiene Socket.IO en el navegador, y un servidor también ha integrado el paquete. Si bien los datos se pueden enviar de varias formas, JSON es el más simple. Resume muchos tipos de transportes, incluidos AJAX y WebSockets, en una sola API. Permite a los desarrolladores enviar y recibir datos sin preocuparse por la compatibilidad entre navegadores \cite{kelleher}.
\vspace{0.8cm}

En cualquier aplicación en tiempo real, mostrar múltiples usuarios en línea es muy importante, esta información debe actualizarse cuando un nuevo usuario se conecta o un usuario en línea se desconecta.
\vspace{0.8cm}

\begin{figure}[H]
  \centering
  \includegraphics[width=1\textwidth]{websocket}
  \caption{Comparación de transporte por sondeo (Ajax) y websockets (Socket.IO).}
\end{figure}

\lstinputlisting[style=ES6, caption=Fragmento de configuración de Socket.IO del sistema]{code/socket.js}

\newpage
\section{Desarrollo de la aplicación web (Frontend)}
El desarrollo web es un campo en constante cambio: la forma en que construimos sitios web hoy en día es completamente diferente de cómo solíamos hacerlo hace un par de años. Con la gran cantidad de herramientas disponibles y las nuevas que aparecen todos los días, la mayoría de las veces los desarrolladores se sienten confundidos sobre qué camino tomar.

\subsection{Patrones de diseño}
Los patrones de diseño facilitan la reutilización de diseños y arquitecturas exitosas. Los patrones de diseño ayudan a elegir alternativas de diseño que hacen que un sistema sea reutilizable y evitar alternativas que comprometan la reutilización. Pueden incluso mejorar la documentación y el mantenimiento de los sistemas existentes.

\subsection{Redux}
Redux es un administrador de estado predecible para aplicaciones JavaScript basado en el patrón de diseño Flux. A medida que una aplicación crece, se hace difícil mantenerla organizada y mantener el flujo de datos. Redux resuelve este problema administrando el estado de la aplicación con un único objeto global llamado Redux Store. Los principios fundamentales de Redux ayudan a mantener la coherencia en toda la aplicación, lo que facilita la depuración y las pruebas. Redux se puede conectar con cualquier biblioteca de JavaScript. Sin embargo, funciona muy bien con ReactJS debido a su naturaleza funcional.
\vspace{0.8cm}

Redux ayuda a separar el estado de la aplicación, crea un almacén global que reside en el nivel superior de una aplicación y alimenta con el estado a todos los componentes internos. A diferencia de Flux, Redux no tiene múltiples objetos de almacenamiento. El estado completo de la aplicación está dentro de un objeto, y potencialmente podría intercambiar la capa de vista con otra biblioteca con el almacenamiento intacto.

\subsubsection{Redux/Flux}
Redux adoptó un gran número de restricciones de la arquitectura Flux: las acciones encapsulan la información para que el Redux Reducer actualice el estado de manera determinista, el estado es un Redux Store \gls{singleton}. El despachador de Flux único se reemplaza con múltiples Redux Reducers pequeños que recogen información de las acciones y la `reducen' a un nuevo estado que luego se guarda en el Redux Store. Cuando se cambia el estado en el Store, la Vista según la suscripción recibe propiedades llamadas props.
\vspace{0.8cm}

\begin{figure}[H]
  \centering
  \includegraphics[width=0.9\textwidth]{flux-redux}
  \caption{Comparación Flux/Redux. (Fuente: Elaboración propia)}
\end{figure}
\vspace{0.8cm}

Para trabajar con Redux se necesitan tres cosas:
\begin{itemize}
  \item Actions (acciones): estos son objetos que deben tener dos propiedades, una que describe el tipo de acción y otra que describe lo que se debe cambiar en el estado de la aplicación.

  \item Reducers (reductores): son funciones que implementan el comportamiento de las acciones. Cambian el estado de la aplicación, en función de la descripción de la acción y la descripción del cambio de estado.

  \item Store (almacén): reúne las acciones y los reducers, manteniendo y cambiando el estado de toda la aplicación; solo hay una store.
\end{itemize}

\subsubsection{Redux Store}
Redux Store contiene un objeto del estado global de la aplicación. Esta actualiza el estado y notifica los componentes suscritos.
\vspace{0.8cm}

\lstinputlisting[style=ES6, caption=Fragmento de código para inicializar el Redux Store]{code/redux-store.js}

\subsubsection{Redux Reducer}
Un Redux Reducer es solo una función pura de JavaScript. Recibe dos parámetros: el estado actual y la acción. Una función pura es aquella que devuelve exactamente la misma salida para la entrada dada. El estado es el objeto Redux Store completo, la acción es el objeto despachado con un tipo requerido y un payload opcional.
\vspace{0.8cm}

\lstinputlisting[style=ES6, caption=Fragmento de código del reducer común de la app]{code/redux-reducer.js}

\subsubsection{Acciones Redux}
La única forma de cambiar el estado es enviando una señal al Store. Esta señal es una acción. Entonces "despachar una acción" significa enviar una señal a Redux Store.
\vspace{0.8cm}

\lstinputlisting[style=ES6, caption=Fragmento de una acción que actualiza el estado]{code/redux-action.js}

Este es un modelo conveniente y directo para estructurar datos en una aplicación y presentarlos en el cliente. La aplicación tiene un estado raíz. Un cambio de estado desencadena actualizaciones de vista. Solo las funciones especiales pueden modificar el estado. Una interacción del usuario activa estas funciones especiales de cambio de estado. Solo se produce un cambio a la vez. Esto significa que el estado central no puede desencadenar ninguna otra acción. Solo una entrada del usuario puede desencadenar otra acción \cite{mukhiya}.

\newpage
\section{Configuración para producción}
Hacer que las aplicaciones de Node.js estén listas para producción es probablemente el tema más oculto y omitido en la literatura de Node.js, pero es uno de los más importantes \cite{mardan}. Se requiere de un método estandarizado para que la efectividad del despliegue en el servidor de producción sea optima.

\subsubsection{Requisitos del servidor}

\begin{itemize}
  \item Servidor CentOS 7 con Git instalado.
  \item Al menos 512 Mb de RAM y 15 Gb de espacio libre en disco.
  \item Acceso de usuario root a través de SSH.
  \item Un nombre de dominio apuntado a la dirección IP del servidor (rosesland.app)
  % \item Editor de texto nano, puede instalarse con este comando:\\
  
  % \code{yum install nano}
\end{itemize}

\subsubsection{Servidor Privado Virtual}
Esta tecnología permite ejecutar las instancias de múltiples servidores en un servidor host físico de forma segura. Es un tipo de servidor de alojamiento web donde el alojamiento generalmente se realiza dividiendo un servidor físico principal en diferentes servidores virtuales múltiples. Cada uno de los servidores del sistema obtiene su propia parte de los recursos en función de los requisitos del cliente.
\vspace{0.8cm}

Este proyecto utiliza un servidor privado virtual (VPS) alojado en Digital Ocean, en donde debe decidir qué sistema operativo se desea, configurar el acceso SSH, crear un firewall, instalar los distintos lenguajes que necesita, etcétera. Para crear el VPS para el proyecto se accede a www.digitalocean.com y se crea una cuenta en Digital Ocean. Es necesario verificar la cuenta con una dirección de correo electrónico antes de poder crear un proyecto. Una vez verificada la dirección de correo electrónico, deberá un método de pago para el servicio. Al finalizar el registro y el modo de financiamiento se puede comenzar a configurar un servidor o Droplet. En Digital Ocean un Droplet es una máquina virtual simple y escalable. Con un costo de solo 5 dolares al mes, el plan básico es lo suficiente para mantener el proyecto en operación.
\vspace{0.8cm}

\begin{figure}[H]
  \centering
  \includegraphics[width=1\textwidth]{digital-ocean}
  \caption{Configuración básica del servidor virtual privado.}
  \label{digital-ocean}
\end{figure}

Una vez que el servidor es creado, se recibe un correo electrónico de Digital Ocean. Este correo electrónico contiene los detalles de inicio de sesión para el servidor, en donde podremos conectarnos mediante SSH.
\vspace{0.8cm}

\subsection{Configurar despliegue automático con Git}
Al utilizar Git, el flujo de trabajo generalmente se dirige solo al control de versiones. Se tiene un repositorio local donde se trabaja y un repositorio remoto donde se mantiene todo sincronizado y se puede trabajar con un equipo y diferentes máquinas. Pero también es posible usar Git para mover una aplicación a producción \cite{vaccaro}.
\vspace{0.8cm}

\subsubsection{Configuración de repositorios}
Para poder empujar cambios a nuestro repositorio remoto se debe establecer la siguiente estructura:

\begin{itemize}
  \item Directorio publico del servidor: /var/www/rosesland.app
  \item Directorio del repositorio del servidor: /var/repo/site.git
\end{itemize}

Se inicia sesión en el VPS desde la consola de comandos y se ingresa lo siguiente:\\
\\
\code{cd /var}\\
\code{mkdir repo \&\& cd repo}\\
\code{mkdir site.git \&\& cd site.git}\\
\code{git init ---bare}\\

\code{---bare} significa que la carpeta no tendrá archivos fuente, solo el control de versiones.

\subsubsection{Git Hooks}
Los repositorios de Git tienen una carpeta llamada \code{hooks}. Esta carpeta contiene algunos archivos de muestra para conectar y realizar acciones personalizadas. La documentación de Git define tres posibles enlaces de servidor: \textit{pre-recepción}, \textit{post-recepción} y \textit{actualización}. Para este proyecto se necesita un `hook' para post-recepción, que se ejecuta cuando un `envío' está completamente terminado.
\vspace{0.8cm}

En el repositorio se encuentran algunos archivos y carpetas, incluida la carpeta \code{hooks}. Se accede a ella mediante el siguiente comando.\\
\\
\code{cd hooks}\\
\\
Se crea el archivo post-recepción escribiendo:\\
\\
\code{cat > post-receive}\\
\\
Al ejecutar este comando, se muestra una línea en blanco que indica que todo lo que escriba se guardará en este archivo. Se debe agregar lo siguiente:\\
\\
\code{\#!/bin/sh}\\
\code{git --work-tree=/var/www/rosesland.app --git-dir=/var/repo/site.git checkout -f}\\
\\
Al terminar, se presiona `control-d' para guardar. Para ejecutar el archivo, se necesita establecer los permisos adecuados usando:\\
\\
\code{chmod +x post-receive}\\
\\
El archivo post-recepción se examina cada vez que se completa un envío y coloca los archivos en \code{/var/www/rosesland.app}, esto permite actualizaciones directas del proyecto facilitando el mantenimiento de codigo y evitando la tarea de tener que conectarse al servidor para cargar los archivos manualmente.
\vspace{0.8cm}

En el repositorio local se necesita configurar la ruta del repositorio remoto. Se le dice Git que agregue un control remoto llamado `live':\\
\\
\code{git remote add live ssh://root@rosesland.app/var/repo/site.git}\\
\\
Con esto es posible empujar cambios en el repositorio al servidor remoto usando el alias `live':\\
\\
\code{git push live master}\\
\\
Con esto se le dice a Git que empuje al remoto `live' en la rama `master'.
\vspace{0.8cm}

\subsubsection{Despliegue Node.js en un entorno de producción}
