\subsection{Configurar despliegue automático con Git}
Al utilizar Git, el flujo de trabajo generalmente se dirige solo al control de versiones. Se tiene un repositorio local donde se trabaja y un repositorio remoto donde se mantiene todo sincronizado y se puede trabajar con un equipo y diferentes máquinas. Pero también es posible usar Git para mover una aplicación a producción.
\vspace{0.8cm}

\subsubsection{Prerrequisitos}

\begin{itemize}
  \item Servidor CentOS 7 con Git instalado.
  \item Al menos 512 Mb de RAM y 15 Gb de espacio libre en disco.
  \item Acceso de usuario root a través de SSH.
  \item Un nombre de dominio apuntado a la dirección IP del servidor.
  \item Editor de texto nano, puede instalarse con este comando:\\

  \code{yum install nano}
\end{itemize}