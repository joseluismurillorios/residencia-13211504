El modelo de E/S impulsado por eventos sin bloqueo le brinda a NodeJS un rendimiento muy atractivo, superando fácilmente los entornos de servidores como PHP y Ruby on Rails, que bloquean las E/S y manejan múltiples usuarios simultáneos en hilos separados para cada uno. Algo importante que se debe saber es que NodeJS no es un \gls{framework} sino un entorno, hay \glspl{framework} que funcionan con Node, como Express y Sails, lo que facilita la creación de aplicaciones.
\vspace{0.8cm}

\begin{figure}[H]
  \centering
  \includegraphics[width=0.8\textwidth]{node-traditional}
  \caption{Comparación de node y servidores tradicionales.}
\end{figure}

Un servidor Node.js tiene un solo subproceso de bucle de eventos (event-loop) que espera E/S en sockets y archivos. Una vez que los datos están listos, activa el método de evento correspondiente y espera hasta que regrese antes de esperar nuevamente por más eventos de E/S. Dado que todas las operaciones de E/S no bloquean, se asegurará de que todo se ejecute correctamente tan pronto como la entrada esté disponible sin ningún bloqueo y sin que se tenga lidiar con problemas de subprocesos múltiples.

\newpage
\subsubsection{Programación basada en eventos}
La filosofía central detrás de NodeJS es la programación basada en eventos. Significa que, el programador,  debe comprender qué eventos están disponibles y cómo responder a ellos. Muchas personas se introducen en la programación basada en eventos mediante la implementación de una interfaz de usuario: el usuario hace clic en algo y se dispara el `evento clic'. Es una buena metáfora, porque se entiende que el programador no tiene control sobre cuándo, o si el usuario va a hacer clic en algo, por lo que la programación basada en eventos es realmente bastante intuitiva \cite{ethan}.
\vspace{0.8cm}

\lstinputlisting[label={node-server}, style=ES6, caption=Configuración servidor NodeJS básico]{code/node-server.js}
En el ejemplo de código \ref{node-server}, el evento es implícito: el evento que se está manejando es una solicitud HTTP. El método http.createServer toma una función como argumento; esta función se invocará cada vez que se realice una solicitud HTTP. El programa simplemente establece el tipo de contenido en texto sin formato y envía la cadena `Hola, mundo!'.
