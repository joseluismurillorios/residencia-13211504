\begin{center}
  {\Large \bf{ABSTRACT}}
\end{center}
This document is aimed at people with intermediate knowledge in programming and a basic reasoning of the operation of the web, here it is described the development of a point of sale system, formed by a server based on Node.js and a web application created with ReactJS , using the most modern tools and procedures for web development.
\vspace{0.8cm}

Point of sale systems are in a demanding environment that requires a server that is accurate and a friendly and responsive user interface. The concrete guidelines of this project answer the question of how to develop a system whose purpose is to manage the sales, manufacturing and logistics processes of a company.
\vspace{0.8cm}

The objective of the project is to integrate the existing online sales services of a company and extend its capabilities, adapting them to the specific requirements of a client, as well as design a web application to help improve the experience in the processes of sales, development and product delivery. The main goal is to implement the system in other companies that requires it.
\vspace{0.8cm}

Below are the procedures necessary to develop the system in question, omitting information that is considered sensitive or that may represent a problem for the company.

\newpage
\begin{center}
  {\Large \bf{RESUMEN}}
\end{center}
Este documento está dirigido a personas con conocimientos intermedios en programación y un razonamiento básico del funcionamiento de la web, aquí se describe el desarrollo de un sistema de punto de venta, formado por un servidor basado en Node.js y una aplicación web creada con ReactJS, utilizando las herramientas y los procedimientos mas modernos para el desarrollo web. 
\vspace{0.8cm}

Los puntos de venta están en un entorno exigente que requiere de un servidor que sea preciso y una interfaz de usuario receptiva y fácil de usar. Las pautas concretas de este proyecto responden a la pregunta de cómo desarrollar un sistema que tiene como propósito el llevar control de los procesos de venta, manufactura y logística de una empresa.
\vspace{0.8cm}

El objetivo del proyecto es integrar los servicios existentes de venta en linea de una compañía y extender su uso, adaptándolos a los requerimientos específicos de un cliente, así como diseñar una aplicación web que permita proporcionar un servicio de venta mas rápido y agilice los proceso de elaboración y entrega de productos, con el fin de poder implementar el sistema en otras empresas que lo requieran.
\vspace{0.8cm}

A continuación se muestran los procedimientos necesarios para elaborar el sistema en cuestión, omitiendo la información que se considera delicada o que pueda representar un problema para la empresa.