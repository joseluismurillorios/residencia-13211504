Las bases de datos NOSQL son una alternativa emergente a las bases de datos relacionales más utilizadas. 
Como su nombre lo indica, no reemplaza completamente a SQL, 
sino que lo complementa de tal manera que puedan coexistir. \\[0.3cm]
El concepto de NOSQL se desarrolló hace mucho tiempo, 
pero fue después de la introducción de la base de datos como servicio (DBaaS) 
que obtuvo un reconocimiento destacado. Debido a la alta escalabilidad proporcionada por NOSQL, 
fue visto como un importante competidor del modelo de base de datos relacional. 
A diferencia de RDBMS, las bases de datos NOSQL están diseñadas para escalar fácilmente a medida que crecen. 
La mayoría de los sistemas NOSQL han eliminado el soporte multiplataforma 
y algunas características adicionales innecesarias de RDBMS, 
haciéndolos mucho más livianos y eficientes que sus contrapartes RDMS.
\subsubsection{Cloud Firestore}
Cloud Firestore es el servicio de base de datos de Google Firebase para aplicaciones móviles. 
Cloud Firestore almacena datos en documentos organizados en colecciones. 
Los datos simples se almacenan en documentos, 
lo cual es fácil y similar a la forma en que se almacenan los datos en JSON. 
Los datos jerárquicos complejos se organizan convenientemente a escala utilizando subcolecciones 
dentro de los documentos. \\[0.3cm]
La escalabilidad es completamente automática, 
lo que significa que no es necesario compartir sus datos en varias instancias. 
Los cargos de Cloud Firestore se basan en las operaciones realizadas en su base de datos 
(lectura, escritura, borrado), ancho de banda y almacenamiento. 
Admite límites de gasto diario para proyectos de Google App Engine, 
para garantizar que no exceda los costos con los que el usuario se sienta cómodo.