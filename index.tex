\documentclass[12pt]{report}

\usepackage{graphicx}
\usepackage[spanish,mexico]{babel}
\usepackage[utf8]{inputenc}
\usepackage{amsmath}
\usepackage{fancyhdr}

\graphicspath{ {./images/} }

\setlength{\oddsidemargin}{0in}
\setlength{\textwidth}{6in}
\setlength{\topmargin}{0in}
\setlength{\voffset}{-0.5in}
\setlength{\hoffset}{0.5in}
\setlength{\textheight}{9in}
\setlength{\textwidth}{6in}
\setlength{\topskip}{0in}
\setlength{\parskip}{2ex}

\title{
   {“Diseño y desarrollo de sistema multi-plataforma para gestión de ventas y logística de microempresa.”}\\
   {\large INSTITUTO TECNOLÓGICO DE TIJUANA}\\
   {\includegraphics{escudo1.png}}
}
\author{José Luis Murillo Ríos}
\date{21 DE JUNIO DEL 2019}


\begin{document}
\setcounter{page}{1}
\pagenumbering{roman}
\thispagestyle{empty}
\begin{center}
   INSTITUTO TECNOLÓGICO DE TIJUANA\\[0.75cm]
   Ingeniería en Sistemas Computacionales\\[0.8in]
\end{center}
\begin{figure}[h]
\begin{center}

\includegraphics[height=5.5 cm]{escudo1.png}
\vspace{0cm}
\end{center}
\end{figure}
\vspace{1cm}
\begin{center}
DISEÑO Y DESARROLLO DE SISTEMA MULTI-PLATAFORMA PARA GESTIÓN DE VENTAS Y LOGÍSTICA DE MICROEMPRESA.\\[4mm]
POR\\[4mm]
JOSÉ LUIS MURILLO RÍOS\\[1cm]
\small PARA OBTENER EL TÍTULO DE INGENIERO EN SISTEMAS COMPUTACIONALES
\vfill
TIJUANA, B.C. \hfill AGOSTO 2019
\end{center}

\newpage
\thispagestyle{empty}
\begin{center}
DISEÑO Y DESARROLLO DE SISTEMA MULTI-PLATAFORMA PARA GESTIÓN DE VENTAS Y LOGÍSTICA DE MICROEMPRESA.\\[1.3cm]
POR\\[0.3cm]
JOSÉ LUIS MURILLO RÍOS\\[0.3cm]
\small AGOSTO 2019\\[0.7cm]
\end{center}

% \chapter*{RESUMEN}
\newpage
\begin{center}
  {\Large \bf{ABSTRACT}}
\end{center}
This document is aimed at people with intermediate knowledge in programming and a basic reasoning of the operation of the web, here it is described the development of a point of sale system, formed by a server based on Node.js and a web application created with ReactJS , using the most modern tools and procedures for web development.
\vspace{0.8cm}

Point of sale systems are in a demanding environment that requires a server that is accurate and a friendly and responsive user interface. The concrete guidelines of this project answer the question of how to develop a system whose purpose is to manage the sales, manufacturing and logistics processes of a company.
\vspace{0.8cm}

The objective of the project is to integrate the existing online sales services of a company and extend its capabilities, adapting them to the specific requirements of a client, as well as design a web application to help improve the experience in the processes of sales, development and product delivery. The main goal is to implement the system in other companies that requires it.
\vspace{0.8cm}

Below are the procedures necessary to develop the system in question, omitting information that is considered sensitive or that may represent a problem for the company.

\newpage
\begin{center}
  {\Large \bf{RESUMEN}}
\end{center}
Este documento está dirigido a personas con conocimientos intermedios en programación y un razonamiento básico del funcionamiento de la web, aquí se describe el desarrollo de un sistema de punto de venta, formado por un servidor basado en Node.js y una aplicación web creada con ReactJS, utilizando las herramientas y los procedimientos mas modernos para el desarrollo web. 
\vspace{0.8cm}

Los puntos de venta están en un entorno exigente que requiere de un servidor que sea preciso y una interfaz de usuario receptiva y fácil de usar. Las pautas concretas de este proyecto responden a la pregunta de cómo desarrollar un sistema que tiene como propósito el llevar control de los procesos de venta, manufactura y logística de una empresa.
\vspace{0.8cm}

El objetivo del proyecto es integrar los servicios existentes de venta en linea de una compañía y extender su uso, adaptándolos a los requerimientos específicos de un cliente, así como diseñar una aplicación web que permita proporcionar un servicio de venta mas rápido y agilice los proceso de elaboración y entrega de productos, con el fin de poder implementar el sistema en otras empresas que lo requieran.
\vspace{0.8cm}

A continuación se muestran los procedimientos necesarios para elaborar el sistema en cuestión, omitiendo la información que se considera delicada o que pueda representar un problema para la empresa.
 
% \chapter*{DEDICATORIA}
\newpage
\begin{center}
  {\Large  \bf{DEDICATORIA}}
\end{center}
\begin{center}
  A mis abuelos, por creer siempre en mi\\
  Gracias
\end{center}

 
% \chapter*{AGRADECIMIENTOS}
\newpage
\begin{center}
  {\Large  \bf{AGRADECIMIENTOS}}
\end{center}
\begin{center}
  Quiero dar gracias a Bolt Media por incluirme en su equipo y darme la oportunidad de crecer.
\end{center}


\newpage
\tableofcontents

\newpage
\listoftables
\addcontentsline{toc}{chapter}{Índice de Tablas} %%% Para introducir una línea en el índice
\listoffigures
\addcontentsline{toc}{chapter}{Índice de Figuras}

\chapter{Introducción}
\thispagestyle{fancy}
\pagenumbering{arabic} %%% esto es para regresar el modo de numeración a numeración arábiga
\setcounter{page}{1}  %%% empezamos en página 1
\thispagestyle{empty}  %%%% la primera página no va enumerada

\section{Antecedentes y definición del problema}
Muchas pequeñas empresas dependen en gran medida al éxito de sus ventas en linea,
por lo que adoptan el uso de tecnologías robustas (como Shopify y WooCommerce) para su administración.
su funcion principal es servir como puntos de entrada para los clientes pero
dichos servicios no estan diseñdos para mostrar informacion detallada a distintas areas de la empresa. \\[0.3cm]
Se solicitó a la empresa Bolt Media Internacional S. de R.L. de C.V.
desarollar una plataforma que conectara las transacciones de rosesland.com,
una plataforma de ventas en linea desarrolada en Shopify a las operaciones internas de la florería,
como lo son las ventas de mostrador, la manofactura y la entrega de los productos. \\[0.3cm]
Actualmente la florería realiza todo este proceso con comandas que los empleados de ventas llenan a mano, para despues
enviarlos al area de elaboración de arreglos florales y posteriormente se entregan al encargado de logistica para
dar indicaciones de la distribucion de los productos. Este proceso de venta es lento y genera una experiencia poco placentera
para los compradores, a su vez el area de manofactura está conformado por artesanos y trabajadores del campo que
tienen un nivel de comprensión de lectura bajo y pueden llegar a cometer errores que retrasen todos los procesos.
Por tales motivos este sistema debe ser fácil de usar para usuarios de distintas areas y a su vez 
el diseño debe de adaptarse tanto a diferentes tamaños de pantallas como a diferentes dispositivos móviles.
Se requiere una base de datos que notifique en tiempo real los cambios
en la información asi como un servidor que administre los permisos adecuados para su manipulación.

\section{Motivación para atenderlos}
Implementar un sistema diseñado principalmente para ser utilizado en pantallas táctiles
que muestre las transacciones realizadas durante el día tanto en mostrador como en la pagina web,
dando un informe detallado de las operaciones de la compañia.

\section{Objetivos generales}
Automatizar el proceso de ventas y entregas de la floreía Rosesland 
e integrar los servicios de su tienda en linea rosesland.com 
mediante una aplicación web progresiva, que sea compatible con los dispositivos existentes de la empresa
y que funcione adecuadamente en navegadores de internet modernos.

\section{Objetivos específicos}
\begin{enumerate}
\item Desarrollar un servidor que adminstre la autenticación de los empleados y su acceso a la información.
\item Implementar una base de datos que notifique a los usuarios cambios en la información en tiempo real.
\item Diseñar una aplicacion web progesiva que garantice una experiencia de usuario óptima para todas las areas de la empresa
\item Integrar los servicios de la tienda en linea rosesland.com con las operaciones internas de la empresa, uniendo los procesos en una sola plataforma
\end{enumerate}

\section{E-commerce (Comercio electrónico)}
El comercio electrónico se refiere al proceso de compra o venta de productos o servicios a través de Internet. 
Las compras en línea se están volviendo cada vez más populares debido a la velocidad y facilidad de uso para los clientes. 
Las actividades de comercio electrónico, como la venta en línea, pueden dirigirse a consumidores u otras empresas. 
% Business to Consumer (B2C) implica la venta en línea de bienes, servicios y el suministro de información directamente a los consumidores. 
% Business to Business (B2B) se refiere al intercambio en línea de productos, servicios o información entre empresas.
Vender en línea puede ayudar a su empresa a llegar a nuevos mercados y aumentar sus ventas e ingresos 
(ya sea a través de su propio sitio web o de un sitio de mercado electrónico).
\subsubsection{Shopify}
Shopify es un servicio web que le permite configurar una tienda en línea para vender sus productos. 
Le da la facilidad organizar sus productos, personalizar el diseño de su tienda, 
aceptar pagos con tarjeta de crédito, rastrear y responder a pedidos. 
Shopify.com permite a los vendedores elegir entre opciones de diseño gratuitas 
o diseños personalizados creados por los usuarios.

\section{Base de datos NoSQL}
Las bases de datos NOSQL son una alternativa emergente a las bases de datos relacionales más utilizadas. 
Como su nombre lo indica, no reemplaza completamente a SQL, 
sino que lo complementa de tal manera que puedan coexistir. \\[0.3cm]
El concepto de NOSQL se desarrolló hace mucho tiempo, 
pero fue después de la introducción de la base de datos como servicio (DBaaS) 
que obtuvo un reconocimiento destacado. Debido a la alta escalabilidad proporcionada por NOSQL, 
fue visto como un importante competidor del modelo de base de datos relacional. 
A diferencia de RDBMS, las bases de datos NOSQL están diseñadas para escalar fácilmente a medida que crecen. 
La mayoría de los sistemas NOSQL han eliminado el soporte multiplataforma 
y algunas características adicionales innecesarias de RDBMS, 
haciéndolos mucho más livianos y eficientes que sus contrapartes RDMS.
\subsubsection{Cloud Firestore}
Cloud Firestore es el servicio de base de datos de Google Firebase para aplicaciones móviles. 
Cloud Firestore almacena datos en documentos organizados en colecciones. 
Los datos simples se almacenan en documentos, 
lo cual es fácil y similar a la forma en que se almacenan los datos en JSON. 
Los datos jerárquicos complejos se organizan convenientemente a escala utilizando subcolecciones 
dentro de los documentos. \\[0.3cm]
La escalabilidad es completamente automática, 
lo que significa que no es necesario compartir sus datos en varias instancias. 
Los cargos de Cloud Firestore se basan en las operaciones realizadas en su base de datos 
(lectura, escritura, borrado), ancho de banda y almacenamiento. 
Admite límites de gasto diario para proyectos de Google App Engine, 
para garantizar que no exceda los costos con los que el usuario se sienta cómodo.

\end{document}